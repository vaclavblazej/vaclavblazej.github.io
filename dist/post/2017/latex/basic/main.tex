\documentclass{article}

\usepackage[utf8]{inputenc} % LaTeX source encoded as UTF-8
\usepackage[czech]{babel} % aby fungovaly CZ znaky

\begin{document}

\section{Vyhledávání v řetězcích}
Řekněme, že vaším úkolem je zjistit, na kterých indexech uvnitř textu se vyskytuje nějaké slovo.
Například pokud máme text $T=abaababcaba$ o délce $t=11$ a $P=ab$ o délce $p=2$, tak výsledkem
 má být $\{0,3,5,8\}$, protož se slovo $ab$ vyskytuje právě na těchto indexech unvnitř $T$.

\subsection{Naivní algoritmus}
Jeden způsob jak vyřešit tento úkol je podívat se na každou pozici, na které se může vzorek vyskytnout,
 a zkusit každou pozici porovnat s textem.
Tento způsob má složitost $O(tp)$, protože pro každou pozici v textu $t$ provedeme maximálně $p$ operací porovnání.

\subsection{KMP algoritmus}
Pro rychlejší vyhledávání v řetězcích se nejčastěji setkáme s algoritmem Morris-Pratt, případně Knuth-Morris-Pratt (mírné vylepšení).
Algoritmus funguje na základě poznatku, že vzorek obrahuje opakující se části tzv. bordery.

Prvně spočítáme border array -- pole, ve kterém jsou uloženy nejdelší bordery prefixu vzorku, které nám pomáhá v běhu vyhledávacího algoritmu.
Potom můžeme za pomoci border array rychle vyhledávat v libovolném textu.

\end{document}

